\documentclass[journal,12pt,twocolumn]{IEEEtran}
\usepackage{setspace}
\usepackage{gensymb}
\usepackage{caption}
%\usepackage{multirow}
%\usepackage{multicolumn}
%\usepackage{subcaption}
%\doublespacing
\singlespacing
\usepackage{csvsimple}
\usepackage{amsmath}
\usepackage{multicol}
%\usepackage{enumerate}
\usepackage{amssymb}
%\usepackage{graphicx}
\usepackage{newfloat}
%\usepackage{syntax}
\usepackage{listings}
\usepackage{iithtlc}
\usepackage{color}
\usepackage{tikz}
\usetikzlibrary{shapes,arrows}



%\usepackage{graphicx}
%\usepackage{amssymb}
%\usepackage{relsize}
%\usepackage[cmex10]{amsmath}
%\usepackage{mathtools}
%\usepackage{amsthm}
%\interdisplaylinepenalty=2500
%\savesymbol{iint}
%\usepackage{txfonts}
%\restoresymbol{TXF}{iint}
%\usepackage{wasysym}
\usepackage{amsthm}
\usepackage{mathrsfs}
\usepackage{txfonts}
\usepackage{stfloats}
\usepackage{cite}
\usepackage{cases}
\usepackage{mathtools}
\usepackage{caption}
\usepackage{enumerate}	
\usepackage{enumitem}
\usepackage{amsmath}
%\usepackage{xtab}
\usepackage{longtable}
\usepackage{multirow}
%\usepackage{algorithm}
%\usepackage{algpseudocode}
\usepackage{enumitem}
\usepackage{mathtools}
\usepackage{hyperref}
%\usepackage[framemethod=tikz]{mdframed}
\usepackage{listings}
    %\usepackage[latin1]{inputenc}                                 %%
    \usepackage{color}                                            %%
    \usepackage{array}                                            %%
    \usepackage{longtable}                                        %%
    \usepackage{calc}                                             %%
    \usepackage{multirow}                                         %%
    \usepackage{hhline}                                           %%
    \usepackage{ifthen}                                           %%
  %optionally (for landscape tables embedded in another document): %%
    \usepackage{lscape}     


\usepackage{url}
\def\UrlBreaks{\do\/\do-}


%\usepackage{stmaryrd}


%\usepackage{wasysym}
%\newcounter{MYtempeqncnt}
\DeclareMathOperator*{\Res}{Res}
%\renewcommand{\baselinestretch}{2}
\renewcommand\thesection{\arabic{section}}
\renewcommand\thesubsection{\thesection.\arabic{subsection}}
\renewcommand\thesubsubsection{\thesubsection.\arabic{subsubsection}}

\renewcommand\thesectiondis{\arabic{section}}
\renewcommand\thesubsectiondis{\thesectiondis.\arabic{subsection}}
\renewcommand\thesubsubsectiondis{\thesubsectiondis.\arabic{subsubsection}}

% correct bad hyphenation here
\hyphenation{op-tical net-works semi-conduc-tor}

%\lstset{
%language=C,
%frame=single, 
%breaklines=true
%}

%\lstset{
	%%basicstyle=\small\ttfamily\bfseries,
	%%numberstyle=\small\ttfamily,
	%language=Octave,
	%backgroundcolor=\color{white},
	%%frame=single,
	%%keywordstyle=\bfseries,
	%%breaklines=true,
	%%showstringspaces=false,
	%%xleftmargin=-10mm,
	%%aboveskip=-1mm,
	%%belowskip=0mm
%}

%\surroundwithmdframed[width=\columnwidth]{lstlisting}
\def\inputGnumericTable{}                                 %%
\lstset{
%language=C,
frame=single, 
breaklines=true,
columns=fullflexible
}
 

\begin{document}
%
\tikzstyle{block} = [rectangle, draw,
    text width=3em, text centered, minimum height=3em]
\tikzstyle{sum} = [draw, circle, node distance=3cm]
\tikzstyle{input} = [coordinate]
\tikzstyle{output} = [coordinate]
\tikzstyle{pinstyle} = [pin edge={to-,thin,black}]

\theoremstyle{definition}
\newtheorem{theorem}{Theorem}[section]
\newtheorem{problem}{Problem}
\newtheorem{proposition}{Proposition}[section]
\newtheorem{lemma}{Lemma}[section]
\newtheorem{corollary}[theorem]{Corollary}
\newtheorem{example}{Example}[section]
\newtheorem{definition}{Definition}[section]
%\newtheorem{algorithm}{Algorithm}[section]
%\newtheorem{cor}{Corollary}
\newcommand{\BEQA}{\begin{eqnarray}}
\newcommand{\EEQA}{\end{eqnarray}}
\newcommand{\define}{\stackrel{\triangle}{=}}

\bibliographystyle{IEEEtran}
%\bibliographystyle{ieeetr}

\providecommand{\nCr}[2]{\,^{#1}C_{#2}} % nCr
\providecommand{\nPr}[2]{\,^{#1}P_{#2}} % nPr
\providecommand{\mbf}{\mathbf}
\providecommand{\pr}[1]{\ensuremath{\Pr\left(#1\right)}}
\providecommand{\qfunc}[1]{\ensuremath{Q\left(#1\right)}}
\providecommand{\sbrak}[1]{\ensuremath{{}\left[#1\right]}}
\providecommand{\lsbrak}[1]{\ensuremath{{}\left[#1\right.}}
\providecommand{\rsbrak}[1]{\ensuremath{{}\left.#1\right]}}
\providecommand{\brak}[1]{\ensuremath{\left(#1\right)}}
\providecommand{\lbrak}[1]{\ensuremath{\left(#1\right.}}
\providecommand{\rbrak}[1]{\ensuremath{\left.#1\right)}}
\providecommand{\cbrak}[1]{\ensuremath{\left\{#1\right\}}}
\providecommand{\lcbrak}[1]{\ensuremath{\left\{#1\right.}}
\providecommand{\rcbrak}[1]{\ensuremath{\left.#1\right\}}}
\theoremstyle{remark}
\newtheorem{rem}{Remark}
\newcommand{\sgn}{\mathop{\mathrm{sgn}}}
\providecommand{\abs}[1]{\left\vert#1\right\vert}
\providecommand{\res}[1]{\Res\displaylimits_{#1}} 
\providecommand{\norm}[1]{\lVert#1\rVert}
\providecommand{\mtx}[1]{\mathbf{#1}}
\providecommand{\mean}[1]{E\left[ #1 \right]}
\providecommand{\fourier}{\overset{\mathcal{F}}{ \rightleftharpoons}}
%\providecommand{\hilbert}{\overset{\mathcal{H}}{ \rightleftharpoons}}
\providecommand{\system}{\overset{\mathcal{H}}{ \longleftrightarrow}}
	%\newcommand{\solution}[2]{\textbf{Solution:}{#1}}
\newcommand{\solution}{\noindent \textbf{Solution: }}
\newcommand{\myvec}[1]{\ensuremath{\begin{pmatrix}#1\end{pmatrix}}}
\providecommand{\dec}[2]{\ensuremath{\overset{#1}{\underset{#2}{\gtrless}}}}
\DeclarePairedDelimiter{\ceil}{\lceil}{\rceil}
%\numberwithin{equation}{subsection}
\numberwithin{equation}{section}
%\numberwithin{problem}{subsection}
%\numberwithin{definition}{subsection}
\makeatletter
\@addtoreset{figure}{section}
\makeatother

\let\StandardTheFigure\thefigure
%\renewcommand{\thefigure}{\theproblem.\arabic{figure}}
\renewcommand{\thefigure}{\thesection}


%\numberwithin{figure}{subsection}

%\numberwithin{equation}{subsection}
%\numberwithin{equation}{section}
%\numberwithin{equation}{problem}
%\numberwithin{problem}{subsection}
\numberwithin{problem}{section}
%%\numberwithin{definition}{subsection}
%\makeatletter
%\@addtoreset{figure}{problem}
%\makeatother
\makeatletter
\@addtoreset{table}{section}
\makeatother

\let\StandardTheFigure\thefigure
\let\StandardTheTable\thetable
\let\vec\mathbf
%%\renewcommand{\thefigure}{\theproblem.\arabic{figure}}
%\renewcommand{\thefigure}{\theproblem}

%%\numberwithin{figure}{section}

%%\numberwithin{figure}{subsection}



\def\putbox#1#2#3{\makebox[0in][l]{\makebox[#1][l]{}\raisebox{\baselineskip}[0in][0in]{\raisebox{#2}[0in][0in]{#3}}}}
     \def\rightbox#1{\makebox[0in][r]{#1}}
     \def\centbox#1{\makebox[0in]{#1}}
     \def\topbox#1{\raisebox{-\baselineskip}[0in][0in]{#1}}
     \def\midbox#1{\raisebox{-0.5\baselineskip}[0in][0in]{#1}}

\vspace{3cm}

\title{ 
	\logo{
BCH Codes
	}
}

\author{ G V V Sharma$^{*}$% <-this % stops a space
	\thanks{*The author is with the Department
		of Electrical Engineering, Indian Institute of Technology, Hyderabad
		502285 India e-mail:  gadepall@iith.ac.in. All content in this manual is released under GNU GPL.  Free and open source.}
	
}	

\maketitle

\tableofcontents

\bigskip

\renewcommand{\thefigure}{\theenumi}
\renewcommand{\thetable}{\theenumi}


\begin{abstract}
	
This manual provides an introduction to BCH codes.
\end{abstract}
%
\section{Generator Polynomial}
\begin{enumerate}[label=\thesection.\arabic*
,ref=\thesection.\theenumi]
\item For a BCH code, the minimal polynomials are given by 
\begin{align}
g_1(x)&=1+x+x^3+x^5+x^{14}\\
g_2(x)&=1+x^6+x^8+x^{11}+x^{14}\\
g_3(x)&=1+x+x^2+x^6+x^9+x^{10}+x^{14}\\
g_4(x)&=1+x^4+x^7+x^8+x^10+x^{12}+x^{14}\\
g_5(x)&=1+x^2+x^4+x^6+x^8+x^9+x^{11}
\nonumber \\
&\,+x^{13}+x^{14}\\
g_6(x)&=1+x^3+x^7+x^8+x^9+x^{13}+x^{14}\\
g_7(x)&=1+x^2+x^5+x^6+x^7+x^{10}+x^{11}
\nonumber \\
&\,+x^{13}+x^{14}\\
g_8(x)&=1+x^5+x^8+x^9+x^{10}+x^{11}+x^{14}\\
g_9(x)&=1+x+x^2+x^3+x^9+x^{10}+x^{14}\\
g_{10}(x)&=1+x^3+x^6+x^9+x^{11}+x^{12}+x^{14}\\
g_{11}(x)&=1+x^4+x^{11}+x^{12}+x^{14}\\
g_{12}(x)&=1+x+x^2+x^3+x^5+x^6+x^7+x^8
\nonumber \\
&\,+x^{10}+x^{13}+x^{14}
\end{align}
Obtain the minimal polynomial matrix.
\\
\solution
\begin{lstlisting}
https://raw.githubusercontent.com/gadepall/EE6317/master/BCH/codes/min_poly_mat.py
\end{lstlisting}
\item Obtain the generator polynomial vector.
\\
\solution The generator polynomial is obtained as
\begin{align}
g(x) =\prod_{i = 1}^{m}g_i(x)
\end{align}
%
where $m=12$.
The following code computes $\vec{g}$. What is the length of $\vec{g}$?
\begin{lstlisting}
https://raw.githubusercontent.com/gadepall/EE6317/master/BCH/codes/gen_poly.py
\end{lstlisting}
\item What is the maximum number of errors that can be corrected by $\vec{g}$?
\\
\solution $m = 12$.
\end{enumerate}
\section{Encoding}
\begin{enumerate}[label=\thesection.\arabic*
,ref=\thesection.\theenumi]
\item Let $\vec{m}$ be a $k \times 1$ message vector and  
\begin{equation}
m(x)=m_{k-1}x^{k-1}+m_{k-2}x^{k-2}+\dots+m_1x+m_0
\end{equation} 
be the  corresponding Message polynomial. 
%\begin{equation}
%g(x)=g_{n-k}x^{n-k}+g_{n-k-1}x^{n-k-1}+\dots+g_1x+g_0
%\end{equation}
%be the Generator polynomial.
%\item Multiply the message polynomial $m(x)$ by $x^{n-k}$, then the message polynomial becomes,
%\begin{equation}
%m(x)x^{n-k}=m_{n-1}x^{n-1}+m_{n-2}x^{n-2}+\dots+m_1x+m_0
%\end{equation}
\item Let 
\begin{equation}
m(x)x^{n-k} = q(x)g(x)+d(x)
\end{equation} 
and 
\begin{equation}
c(x)=m(x)x^{n-k}+d(x)
\end{equation}
%
\item If $k = 3072$ find the length of $\vec{c}$.
\item 
Write a program to compute the corresponding coefficient vector $\vec{c}$. This is the output of the BCH 
encoder.
\\
\solution
\begin{lstlisting}
https://raw.githubusercontent.com/gadepall/EE6317/master/BCH/codes/encoder.py
\end{lstlisting}
\end{enumerate}



\section{Appendix}

\section{Berlekamp's Decoding Algorithm}
\begin{enumerate}[label=\thesection.\arabic*
,ref=\thesection.\theenumi]
\item Define the syndrome polynomial $S(x)$
\item Intialization : $k=0, \Lambda^{(0)}(x)=1,T^{(0)}=1$
\item Let $\Delta^{(2k)}$ be the coefficient of $x^{2k+1}$ in $\Lambda^{(2k)}[1+S(x)]$ 
\item Compute  
\begin{equation}
\Delta^{(2k+2)}(x)=\Lambda^{(2k)}(x) +\Delta^{(2k)}[x.T^{(2k)}(x)]
\end{equation}
\item Compute \begin{equation}
    	T^{(2k+2)}(x) =
    \begin{cases}
    	x^2T^{(2k)}(x) & \text{if $\Delta^{(2k)}=0$ or deg}\sbrak{\Lambda^{(2k)}(x)}>k\\
      \frac{x \Lambda^{(2k)}(x)}{\Delta^{(2k)}} & \text{if $\Delta^{(2k)}\neq 0$ or 
deg}\sbrak{\Lambda^{(2k)}(x)}\leq k
    \end{cases}
  \end{equation}
\item Set $k=k+1$.If $k<t$ then go to step3.
\item Return the Error Locator polynomial $\Lambda(x)=\Lambda^{(2k)}(x)$
\end{enumerate}
\section{The Chien's Search Algorithm}
\begin{enumerate}[label=\thesection.\arabic*
,ref=\thesection.\theenumi]
\item Take $\alpha ^j$ as test root . $0\leq j \leq n-1$.
\item if $\Lambda_i $ test every root and if its equals to zero. Then that is root.
\item  Flip the bit values at root positions.

\item Let the Received polynomial be $r(x)$ i.e which contains both transmitted codeword polynomial $c(x)$ and 
the 
error polynomial $e(x)$ \begin{equation}
e(x)=e_0+e_1x^1+...+e_{n-1}x^{n-1}
\end{equation} Where $e_i$ represents the value of the error at the location. For binary BCH codes  $e_i$ is 
either 0 or 1.
\begin{equation}
r(x)=c(x)+e(x)=r_{n-1}x^{n-1}+r_{n-2}x^{n-2}+\dots+r_1x+r_0
\end{equation}
Definie, Syndrome \begin{equation}
S_i=r(\alpha^i)=c(\alpha^i)+e(\alpha^i)=e(\alpha^i)
\end{equation} Where $\alpha^i $ is a root of the codeword.

Suppose that $v$ errors occurred, and $0\leq v \leq t$.
Let the error occurs at $i_1,i_2,\dots,i_v$. 




The Decoding Process, for a t-error correcting code will follows the basic steps,
\item Compute the Syndrome $S=(S_1,S_2,\dots,S_{2t})$ from the received polynomial $r(x)$
\item Determine the error-location polynomial $\sigma(x)$ from the syndrome components $S_1,S_2,\dots,S_{2t}$ 
using the Berlekamp's Algorithm.
\item Using the Chain searching algorithm, determine the error-locations by finding the roots of $\sigma(x)$, 
the flip the posisions in $r(x)$. Which is the estimated message vector polynomial $\hat{c}(x)$ .

\item 
where $\vec{w}$ is $p \times 1$ and $\vec{X}$ is $N \times p$.  Show that
\begin{align}
E\brak{\hat{\vec{w}}} = \vec{w}
\end{align}
%
\item If the covariance matrix of $\vec{y}$ is
\begin{align}
\vec{C}_{\vec{y}} = \sigma^2\vec{I}
\end{align}
%E\sbrak{\hat{\vec{w}}\hat{\vec{w}}^T}= 
%\sigma^2\brak{\vec{X}^T\vec{X}}^{-1}
show that
\begin{align}
\vec{C}_{\vec{w}} = \sigma^2\brak{\vec{X}^T\vec{X}}^{-1}
%E\sbrak{\hat{\vec{w}}\hat{\vec{w}}^T}= 
\end{align}
%
\item Let
%
\begin{align}
\hat{\vec{y}} &= \vec{X}\hat{\vec{w}}
\\
\hat{\sigma}^2 &= \frac{1}{N-p}\norm{\vec{y}-\hat{\vec{y}}}^2
\\
\vec{y}-\hat{\vec{y}} &\sim \mathcal{N}\brak{\vec{0}, \sigma^2\vec{I}}
\end{align}
%
Show that
\begin{align}
\brak{N-p}\hat{\sigma}^2 \sim \sigma^2\chi_{N-p}^2
\end{align}
\item Let 
\begin{align}
\hat{z}=\frac{w_j}{\hat{\sigma}\sqrt{v_j}}
\end{align}
%
where $v_j$ is the diagonal element of $\brak{\vec{X}^T\vec{X}}^{-1}$.  
If $w_j= 0$,  show 
that $z_j$ has a $t_{N-p}$ distribution.
\item Plot $\pr{\abs{Z}>z}$ for $t_{30}, t_{100}$ and the standard normal 
distribution.
\end{enumerate}
\section{Applications}
\begin{enumerate}[label=\thesection.\arabic*
,ref=\thesection.\theenumi]
\item Explain how \eqref{eq:opt_mse} can be used to obtain the  Nearest 
Neighbour approximation.
\item Repeat the exercise for the least squares method.
\end{enumerate}
%
\end{document}
